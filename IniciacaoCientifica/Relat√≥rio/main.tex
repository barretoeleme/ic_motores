\documentclass[12pt]{article}

% Pacotes
\usepackage[utf8]{inputenc}
\usepackage[T1]{fontenc}
\usepackage[brazil]{babel}
\usepackage{graphicx}
\usepackage{amsmath, amssymb}
\usepackage{geometry}
\usepackage{setspace}
\usepackage{indentfirst}
\usepackage{hyperref}
\usepackage{caption}
\usepackage{float}
\usepackage{enumitem}
\usepackage[style=abnt, backend=biber]{biblatex}
\addbibresource{referencias.bib} % Arquivo .bib com referências

% Configurações de página
\geometry{a4paper, top=3cm, bottom=3cm, left=3cm, right=2.5cm}
\setlength{\parindent}{1.25cm}
\setlength{\parskip}{0.2cm}
\onehalfspacing

% Início do documento
\begin{document}

% Capa
\begin{titlepage}
    \centering
    \vspace*{3cm}
    {\Large\textbf{Universidade Federal Exemplar}}\\[0.3cm]
    {\large Departamento de Ciência da Computação}\\[3cm]
    
    {\huge\bfseries Modelos básicos para IPMSMs}\\[1.5cm]
    
    {\large Autor: Nome Completo}\\[0.5cm]
    {\large Matrícula: 123456789}\\[0.5cm]
    {\large Professor: Nome do Professor}\\[4cm]
    
    {\large Cidade – Estado}\\
    {\large \today}
\end{titlepage}

% Sumário
\tableofcontents
\newpage

% Introdução
\section{Introdução}
% O que são IPMSMs?
% Por quê é importante estudas IPMSMs?
% Por quê usar saprendizado de máquina para a análise de IPSMs?
% Objetivo do trabalho: Encontrar o melhor modelo possível para a predição de loss das IPMSMs analisadas.


% Metodologia
\section{Fundamentos}
% Descreva os métodos, algoritmos ou procedimentos utilizados. Pode incluir diagramas, equações ou pseudocódigo.

\subsection{Motores analisados}

\subsection{Bases de dados}

\subsection{Modelos Utilizados}
\subsubsection{Regressão Linear}
\subsubsection{Árvores de Regressão}
\subsubsection{Random Forests}
\subsubsection{XGboos}
\subsubsection{CatBoost}

\section{Metodologia}
Para atingir os objetivos deste trabalho, definimos 2 experimentos \dots

\subsection{Experimento 1: Identificar os Modelos mais Promissores}

\subsection{Experimento 2: Identificar o mlehor conjunto de hiper parâmetros para os modelos mais promissores}


% Resultados
\section{Resultados}
Apresente os resultados obtidos (tabelas, gráficos, análises). Interprete os dados de forma clara e objetiva.

% Conclusão
\section{Conclusão}
Retome os objetivos e destaque os principais achados. Comente limitações e sugestões para trabalhos futuros.

% Referências
\newpage
\printbibliography

\end{document}
